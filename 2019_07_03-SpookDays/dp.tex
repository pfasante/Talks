\section{Division Property}
\begin{frame}[plain]
    \sectionpage
    \vfill
\end{frame}

\begin{frame}{Division Property}
    \begin{block}{Main Idea: Division Property}
        \begin{itemize}
            \item Generalisation of Integral and Higher Order Differential attacks
            \item Captures properties of bits in a set
            \item For standard integral attacks: zero-sum, all or constant
            \item The Division Property allows to capture properties \enquote{in between} these\\
                  (even if they do not have such a nice description as \eg/ the zero-sum)
        \end{itemize}
    \end{block}
    \begin{block}{Division Property}
        ???
    \end{block}
\end{frame}

\begin{frame}{Division Property}{Related Work}
    \begin{timeline}
        \Task[\cite{EC:Todo15}]{Publication of DP attack}
        \Task[\cite{C:Todo15}]{DP attack breaking\\ full \misty/}
        \Task[\cite{C:BouCan16}]{Analysis of DP, S-box properties to resist\\ this attack}
        \Task[\cite{FSE:TodMor16}]{Bit-based DP}
        \Task[\cite{AC:XZBL16}]{MILP based search of\\ DP distinguishers}
        \Task[%
                \cite{C:TIHM17}\\
                \cite{C:WHTLIM18}
            ]{DP based Cube attacks on stream ciphers}
    \end{timeline}
\end{frame}

\begin{frame}{Division Trails}
    \begin{block}{Division Trail}
        ???
    \end{block}
    \begin{block}{Propagating Bit-Based Division Trails}
        \vspace*{5pt}
        \begin{columns}
        \begin{column}{0.3\textwidth}
            \vspace*{-10pt}
            \begin{gather*}
                \mathrm{copy} : x \mapsto (x, x) \\
                \mathcal{D}^1_x \stackrel{\mathrm{copy}}{\to} \begin{cases*}
                    \mathcal{D}^1_{(0,0)}       & if $x = 0$ \\
                    \mathcal{D}^1_{(0,1),(1,0)} & if $x = 1$
                \end{cases*}
            \end{gather*}
            \vspace*{-3pt}
        \end{column}
        \begin{column}{0.3\textwidth}
            \vspace*{-10pt}
            \begin{gather*}
                \mathrm{xor} : (x,y) \mapsto x + y \\
                \mathcal{D}^{1,2}_{(k_0,k_1)} \stackrel{\mathrm{xor}}{\to} \mathcal{D}^1_{k_0+k_1}
            \end{gather*}
        \end{column}
        \begin{column}{0.3\textwidth}
            \centering
            S-box $S : \F_2^n \to \F_2^n$:\\
            see~\cite[Algorithm~2]{AC:XZBL16},
            computes for all $u \in \F_2^n$\\
            \vspace*{-15pt}
            \begin{equation*}
                \mathcal{D}^{1,n}_{u} \stackrel{S}{\to} \mathcal{D}^{1,n}_V
            \end{equation*}
            \st/ $u \to v$ is a DT $\forall v \in V$.
        \end{column}
        \end{columns}
        \vspace*{5pt}
    \end{block}
\end{frame}

\begin{frame}{Division Property}
    \begin{block}{\textbf{Goal}: Apply security argument from}
    \begin{quote}
        \fullcite{AC:XZBL16}.
    \end{quote}
    \end{block}
    \begin{exampleblock}{What do we get from this?}
        \centering
        Number of rounds for which a division property/integral distinguisher exists.
    \end{exampleblock}
    \begin{block}{Approach (similiar to Subspace Trails)}
        \begin{itemize}
            \item Pick starting DPs in a way that covers all possibilities
            \item Model division trail propagations as MILP
            \item Find solutions for this over increasing number of rounds
        \end{itemize}
    \end{block}
\end{frame}

\frame<1-3>[label=dp-milp-model]
{
\frametitle{Division Property}
\framesubtitle{MILP model}
    \vfill
    \begin{columns}
        \begin{column}{0.475\textwidth}
            \begin{block}{Mixed Integer Linear Programs}
                \centering
                Typical description of a MILP\\[10pt]
                \begin{tabular}{cccc}
                    \toprule
                    Objective           & & $\max$/$\min$ & $c^\top x$       \\
                    linear inequalities & & \textrm{subject to} & $Ax \leqslant b$ \\
                    \bottomrule
                \end{tabular}
                \vspace{10pt}
                \begin{itemize}
                    \item $A, b, c$ known coefficients
                    \item $x$ unknown variables
                \end{itemize}
            \end{block}
        \end{column}
        \pause
        \begin{column}{0.475\textwidth}
            \begin{block}{Applying MILPs to find Division Properties}
                \textbf{Goal}: Model Division Property as a MILP\\[6.5pt]

                We need:
                \begin{itemize}
                    \colorize<4> \item Objective function
                    \colorize<4> \item Starting DP
                    \colorize<3> \item Propagation Rules
                    \colorize<4> \item Stopping Rule
                \end{itemize}
            \end{block}
        \end{column}
    \end{columns}
    \vfill
    \visible<5->{%
        \centering
        Using this, we can now model the DP search for \clyde/
    }
    \vfill
}

\begin{frame}{Division Property}{Modeling Propagation Rules: $\mathrm{copy}$}
    \centering
    Based on eprint's \href{https://ia.cr/2016/392}{2016/392}, \href{https://ia.cr/2016/811}{2016/811}, and \href{https://ia.cr/2016/1101}{2016/1101}
    \begin{columns}
        \begin{column}{0.35\textwidth}
            \begin{block}{Propagation Rule}
                \vspace*{-15pt}
                \begin{gather*}
                    \mathrm{copy} : x \mapsto (x, x) \\
                    \mathcal{D}^1_x \stackrel{\mathrm{copy}}{\to} \begin{cases*}
                        \mathcal{D}^1_{(0,0)}       & if $x = 0$ \\
                        \mathcal{D}^1_{(0,1),(1,0)} & if $x = 1$
                    \end{cases*}
                \end{gather*}
                \vspace*{-3pt}
            \end{block}
            \begin{block}{Valid Transitions}
                \begin{enumerate}
                    \item \quad $(0) \stackrel{\mathrm{copy}}{\to} (0, 0)$
                    \item \quad $(1) \stackrel{\mathrm{copy}}{\to} (0, 1)$
                    \item \quad $(1) \stackrel{\mathrm{copy}}{\to} (1, 0)$
                \end{enumerate}
            \end{block}
            \pause
        \end{column}
        \begin{column}{0.575\textwidth}
            \begin{block}{MILP Model}
                \begin{itemize}
                    \item Given division trail $(x) \stackrel{\mathrm{copy}}{\to} (y, z)$\\[2pt]
                    \item Propagation represented by the (in)equality
                        \begin{gather*}
                            x - y - z = 0\\
                            x, y, z \in \set{0,1}
                        \end{gather*}
                \end{itemize}
            \end{block}
        \end{column}
    \end{columns}
\end{frame}

\begin{frame}{Division Property}{Modeling Propagation Rules: $\mathrm{xor}$}
    \centering
    Based on eprint's \href{https://ia.cr/2016/392}{2016/392}, \href{https://ia.cr/2016/811}{2016/811}, and \href{https://ia.cr/2016/1101}{2016/1101}
    \begin{columns}
        \begin{column}{0.35\textwidth}
            \begin{block}{Propagation Rule}
            \vspace*{-10pt}
            \begin{gather*}
                \mathrm{xor} : (x,y) \mapsto x + y \\
                \mathcal{D}^{1,2}_{(k_0,k_1)} \stackrel{\mathrm{xor}}{\to} \mathcal{D}^1_{k_0+k_1}
            \end{gather*}
            \end{block}
            \begin{block}{Valid Transitions}
                \begin{enumerate}
                    \item \quad $(0,0) \stackrel{\mathrm{xor}}{\to} (0)$
                    \item \quad $(1,0) \stackrel{\mathrm{xor}}{\to} (1)$
                    \item \quad $(0,1) \stackrel{\mathrm{xor}}{\to} (1)$
                \end{enumerate}
            \end{block}
            \pause
        \end{column}
        \begin{column}{0.575\textwidth}
            \begin{block}{MILP Model}
            \begin{itemize}
                \item Given division trail $(x, y) \stackrel{\mathrm{xor}}{\to} (z)$\\[2pt]
                \item Propagation represented by the (in)equality:
                    \begin{gather*}
                        x + y - z = 0\\
                        x, y, z \in \set{0,1}
                    \end{gather*}
            \end{itemize}
            \end{block}
        \end{column}
    \end{columns}
\end{frame}

\begin{frame}{Division Property}{Modeling Propagation Rules: S-box}
    \centering
    Based on approach by \textcite{AC:SHWQMS14} for differential case
    \begin{columns}
        \begin{column}{0.35\textwidth}
            \begin{block}{Propagation Rule}
                \centering
                S-box $S : \F_2^n \to \F_2^n$:\\
                see~\cite[Algorithm~2]{AC:XZBL16},
                computes for all $u \in \F_2^n$\\
                \vspace*{-15pt}
                \begin{equation*}
                    \mathcal{D}^{1,n}_{u} \stackrel{S}{\to} \mathcal{D}^{1,n}_V
                \end{equation*}
            \end{block}
            \begin{block}{Valid Transitions}
                \begin{columns}
                    \begin{column}{0.5\linewidth}
                        \begin{enumerate}
                            \item \quad $u \stackrel{S}{\to} v_1$
                            \item[] \quad \hspace{4pt} $\cdots$
                            \item[\tikz{\node[scale=0.8,minimum width=2pt,minimum height=2pt,color=white,fill=saphierblau]{\scriptsize k}}] \quad $u \stackrel{S}{\to} v_k$
                        \end{enumerate}
                    \end{column}
                    \begin{column}{0.5\linewidth}
                        for $v_i \in V$
                    \end{column}
                \end{columns}
            \end{block}
            \pause
        \end{column}
        \begin{column}{0.575\textwidth}
            \begin{block}{MILP Model}
            \begin{itemize}
                \item Interpret set of all valid $(u, v) \in \F_2^{2n}$ as polyhedron
                \item Get inequalities from its H-representation
                \item Choose inequalities for model by
                    \begin{itemize}
                        \item Greedy Approach~\cite{AC:SHWQMS14}
                        \item MILP Approach~\cite{SecITC:SasTod17} (seems to be slower)
                    \end{itemize}
            \end{itemize}
            \end{block}
        \end{column}
    \end{columns}
\end{frame}

\againframe<3-4>{dp-milp-model}

\begin{frame}{Division Property}{Objective, Start, Stop}
    \begin{block}{What are we looking for?}
        \begin{itemize}
            \item Unit vectors in output division property correspond to unbalanced bits.
            \item We have to exclude these from our MILP model.
            \item When minimising the sum over the output variables, we find these unit vectors first.
        \end{itemize}
    \end{block}
    \begin{block}{Objective}
        \begin{equation*}
            \mathrm{minimise} \quad x_0^r + x_1^r + \cdots + x_n^r
        \end{equation*}
    \end{block}
\end{frame}

\begin{frame}{Division Property}{Objective, Start, Stop}
\end{frame}

\begin{frame}{Division Property}{Objective, Start, Stop}
    \begin{columns}
        \begin{column}{0.575\textwidth}
            \begin{block}{Model Stopping Rule}
                \begin{algorithmic}[1]
                    \Require{A Division Property MILP model $\mathcal{M}$}
                    \Ensure{A distinguisher exists or not}
                    \Statex{}
                    \Function{DP Distinguisher Search}{$\mathcal{M}$}
                    \While{$\mathcal{M}$ has feasible solution}
                        \State{}Solve $\mathcal{M}$
                        \visible<2->{%
                        \If{objective value equals one}
                            \State{}Let $v$ be the variable $=1$ for solution
                            \State{}Add constraint $v = 0$ to $\mathcal{M}$
                        \visible<3->{%
                        \Else{}
                            \State{}\Return{Found distinguisher}
                        }
                        \EndIf{}
                        }
                    \EndWhile{}
                    \visible<3->{%
                    \State{}\Return{No distinguisher exists}
                    }
                    \EndFunction{}
                \end{algorithmic}
            \end{block}
        \end{column}
        \begin{column}{0.4\textwidth}
            \begin{block}{Stopping Rule}
                \begin{itemize}
                    \item<2-> Unit vectors in output division property correspond to unbalanced bits.
                    \item<2-> We have to exclude these from our MILP model.
                    \item<3-> If no more unit vectors where found, but MILP still has feasible solution, a distinguisher exists.
                \end{itemize}
            \end{block}
        \end{column}
    \end{columns}
\end{frame}

\againframe<4-5>{dp-milp-model}

\begin{frame}{Division Property}{Results}
    \begin{block}{Division Property distinguisher for \clyde/}
        \begin{itemize}
            \item 8 Rounds
        \end{itemize}
    \end{block}
\end{frame}
