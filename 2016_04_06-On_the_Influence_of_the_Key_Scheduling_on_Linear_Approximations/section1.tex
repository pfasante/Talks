\section{Motivation}
\begin{frame}{Assumptions made in Block Cipher Designs}{Motivation}
	\begin{columns}
		\begin{column}{0.48\textwidth}
			\begin{block}{Independent Round Keys and Key Schedule Behaviour}
				\centering
				\vspace{1em}
				\includegraphics[width=\maxwidth{\textwidth},
				height=\maxheight{.55\textheight},
				keepaspectratio]%
				{data/plots/ssaes.pdf}
			\end{block}
		\end{column}
		\begin{column}{0.5\textwidth}
			\begin{block}{Hypothesis of Stochastic Equivalence}
				\vspace{1em}
				Cipher behaves the same when instantiated with
				\begin{itemize}
					\item independent round keys, or
					\item round keys generated by key schedule.
				\end{itemize}
				\vspace{1em}
			\end{block}
		\end{column}
	\end{columns}

%	\begin{columns}
%		\begin{column}{0.48\textwidth}
%			\begin{block}{}
%				\centering
%				\includegraphics[width=\maxwidth{\textwidth},
%				height=\maxheight{.55\textheight},
%				keepaspectratio]%
%				{data/plots/smallpresent4.pdf}
%			\end{block}
%		\end{column}
%		\begin{column}{0.5\textwidth}
%			\begin{itemize}
%				\item guessing a wrong round key randomises our distingiushing value
%			\end{itemize}
%		\end{column}
%	\end{columns}
\end{frame}
